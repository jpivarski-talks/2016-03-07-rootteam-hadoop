\documentclass{beamer}
%
% Choose how your presentation looks.
%
% For more themes, color themes and font themes, see:
% http://deic.uab.es/~iblanes/beamer_gallery/index_by_theme.html
%
\mode<presentation>
{
  \usetheme{default}      % or try Darmstadt, Madrid, Warsaw, ...
  \usecolortheme{default} % or try albatross, beaver, crane, ...
  \usefonttheme{default}  % or try serif, structurebold, ...
  \setbeamertemplate{navigation symbols}{}
  \setbeamertemplate{caption}[numbered]
  \setbeamertemplate{footline}[frame number]
} 

\usepackage[english]{babel}
\usepackage[utf8x]{inputenc}

\title[2016-03-07-rootteam-hadoop]{Computing practices observed in industry}
\author{Jim Pivarski}
\institute{Princeton University -- DIANA}
\date{March 7, 2016}

\xdefinecolor{darkblue}{rgb}{0.1,0.1,0.7}

\begin{document}

\begin{frame}
  \titlepage
\end{frame}

\begin{frame}{Context}

\begin{block}{}
The commercial ``Big Data'' movement developed more or less independently of high energy physics, even though many of the same problems had to be solved.
\end{block}

\begin{block}{}
This talk is about the differences I have seen between the two communities, with an emphasis on technical choices, to aid in integration and interoperablity.
\end{block}
\end{frame}

\begin{frame}{Language choice}
\begin{block}{Short answer}
Commercial distributed systems are usually implemented in Java, and so distributed data processing systems like Hadoop and Spark are also on the Java Virtual Machine (JVM).
\end{block}

\begin{block}{Long answer}
This is the topic I get the most questions about, so I'll break it down by project (next page).
\end{block}
\end{frame}

\begin{frame}{Language choice}
\begin{itemize}
\item Credit card company: SAS and Python ({\it pure} Python, not even Numpy).

\item Web advertising start-up: Python.

\item NASA (open source Project Matsu): Java simply because it was a new project using Hadoop and HBase. I think they ordinarily use C++ for image processing.

\item Monitoring auto traffic: Storm real-time analysis (which is in Clojure, a JVM language, but I wrote my code in Scala).

\item Auto insurance: SQL over Hadoop, using Hive and Pig. User-defined functions were in Java because Hive and Pig (and Hadoop) are Java.

\item Military project: extremely Java-centric.

\item Data science start-up: most data analyses in R, a little in Python, but the web-facing data backbone was strictly Java.
\end{itemize}
\end{frame}

%% \hspace{-0.83 cm} \textcolor{darkblue}{\Large Apache Hadoop}

\begin{frame}{Major frameworks}
\begin{block}{Apache Hadoop}
Framework for performing map-reduce calculations. Used as a foundation for a big data cluster because of\ldots
\begin{itemize}
\item the HDFS distributed filesystem (even variants like MapR, which don't use HDFS, use its API),
\item its suite of InputFormats that split files by logical records,
\item ZooKeeper, which coordinates job configuration and synchronization across a distributed service.
\end{itemize}
\end{block}

\begin{block}{Apache Spark}
Generalizes from map-reduce to arbitrary DAG data pipelines, optimized for iterative procedures, with an interactive prompt. May be used on any cluster manager, but usually Hadoop.
\begin{itemize}
\item User interfaces: native Scala, Java, Python (through sockets), and R (through pipes).
\end{itemize}
\end{block}
\end{frame}

\begin{frame}{Data pipelines and databases (frequently encountered)}

It's not uncommon to use several frameworks in a single project, whether they're orthogonal in purpose or not.

\begin{description}
\item[Apache Storm] real-time analysis, a fault-tolerant data pipeline.
\item[Apache Drill] rapid response to queries (which I think would be ideal for plotting).
\item[Apache HBase] random-access tabular database over Hadoop.
\item[Apache Hive] SQL over Hadoop ({\it not} random-access).
\item[Apache Pig] custom language (Pig Latin), ``eats any data format.''
\item[Apache Mesos] cluster manager (using ZooKeeper).
\item[Apache Kafka] message queue.
\item[Apache Flume] queues for log files.
\item[ElasticSearch] full-text search engine.
\item[MongoDB] indexable JSON document store.
\end{description}
\end{frame}

\begin{frame}{Data formats (frequently encountered)}

It's also not uncommon to mix file formats and use a lot of text-based formats. JSON was perhaps 80\% of what I saw.

\begin{description}
\item[CSV] table of primitives: numbers, booleans, strings (text).
\item[JSON] arrays and maps of primitives (text).
\item[XML] structures with an optional schema (text).
\item[Apache Avro] JSON-like binary format with algebraic data types (arrays, maps, records, and unions of primitives). Similar to \textcolor{darkblue}{Thrift} and \textcolor{darkblue}{Protocol buffers}.
\item[Parquet] similar to Avro, but stored column-wise for speed.
\item[Sequence files] structured container of arbitrary binary blobs intended as splitting hints for Hadoop.
\end{description}

And many application-specific formats.
\end{frame}


%% Mahout
%% MLLib

%% Numpy, SciPy, SciKit-Learn


\end{document}
